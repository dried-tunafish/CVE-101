\documentclass[11pt]{article}
%\usepackage{geometry}
%\geometry{papersize={8.5in,13in},total={4.8in,6.8in}}
\usepackage[inner=1.5cm,outer=1.5cm,top=2.5cm,bottom=2.5cm,paperwidth=8.5in,paperheight=13in]{geometry}

\pagestyle{empty}
\usepackage{graphicx}
\usepackage{fancyhdr, lastpage, bbding, pmboxdraw}
\usepackage[usenames,dvipsnames]{color}
\definecolor{darkblue}{rgb}{0,0,.6}
\definecolor{darkred}{rgb}{.7,0,0}
\definecolor{darkgreen}{rgb}{0,.6,0}
\definecolor{red}{rgb}{.98,0,0}
\usepackage[colorlinks,pagebackref,pdfusetitle,urlcolor=darkblue,citecolor=darkblue,linkcolor=darkred,bookmarksnumbered,plainpages=false]{hyperref}
\renewcommand{\thefootnote}{\fnsymbol{footnote}}

\pagestyle{fancyplain}
\fancyhf{}
\lhead{ \fancyplain{}{CVE101 - Civil Engineering Orientation} }
%\chead{ \fancyplain{}{} }
\rhead{ \fancyplain{}{\today} }
%\rfoot{\fancyplain{}{page \thepage\ of \pageref{LastPage}}}
\fancyfoot[RO, LE] {page \thepage\ of \pageref{LastPage} }
\thispagestyle{plain}

%%%%%%%%%%%% LISTING %%%
\usepackage{listings}
\usepackage{caption}
\DeclareCaptionFont{white}{\color{white}}
\DeclareCaptionFormat{listing}{\colorbox{gray}{\parbox{\textwidth}{#1#2#3}}}
\captionsetup[lstlisting]{format=listing,labelfont=white,textfont=white}
\usepackage{verbatim} % used to display code
\usepackage{fancyvrb}
\usepackage{acronym}
\usepackage{amsthm}
\VerbatimFootnotes % Required, otherwise verbatim does not work in footnotes!



\definecolor{OliveGreen}{cmyk}{0.64,0,0.95,0.40}
\definecolor{CadetBlue}{cmyk}{0.62,0.57,0.23,0}
\definecolor{lightlightgray}{gray}{0.93}



\lstset{
%language=bash,                          % Code langugage
basicstyle=\ttfamily,                   % Code font, Examples: \footnotesize, \ttfamily
keywordstyle=\color{OliveGreen},        % Keywords font ('*' = uppercase)
commentstyle=\color{gray},              % Comments font
numbers=left,                           % Line nums position
numberstyle=\tiny,                      % Line-numbers fonts
stepnumber=1,                           % Step between two line-numbers
numbersep=5pt,                          % How far are line-numbers from code
backgroundcolor=\color{lightlightgray}, % Choose background color
frame=none,                             % A frame around the code
tabsize=2,                              % Default tab size
captionpos=t,                           % Caption-position = bottom
breaklines=true,                        % Automatic line breaking?
breakatwhitespace=false,                % Automatic breaks only at whitespace?
showspaces=false,                       % Dont make spaces visible
showtabs=false,                         % Dont make tabls visible
columns=flexible,                       % Column format
morekeywords={__global__, __device__},  % CUDA specific keywords
}

%%%%%%%%%%%%%%%%%%%%%%%%%%%%%%%%%%%%
\begin{document}
\begin{center}
{\Large \textsc{CVE101 - CIVIL ENGINEERING ORIENTATION}}
\end{center}
\begin{center}
1st Semester, 2023-2024
\end{center}
%\date{September 26, 2014}

\begin{center}
\rule{6in}{0.4pt}
\begin{minipage}[t]{.75\textwidth}
\begin{tabular}{llcccll}
\textbf{Instructor:} & Engr. Nikol O. Telen & & &  & \textbf{Time:} & F 14:00 -- 17:00 \\
\textbf{Email:} &  \href{mailto:nikol.telen@msugensan.edu.ph}{nikol.telen@msugensan.edu.ph} & & & & \textbf{Place:} & H2-01
\end{tabular}
\end{minipage}
\rule{6in}{0.4pt}
\end{center}
\vspace{.5cm}
\setlength{\unitlength}{1in}
\renewcommand{\arraystretch}{2}

\begin{tabular}{lrlccll}
\textbf{Hours per Week} & : &2 Hours Lecture & &  &\\
\textbf{Prerequisite} & : & N/A & & &\\
\textbf{Credits} & : & 2 & & &
\end{tabular}
\vskip.15in

\noindent\textbf{Course Pages:} \begin{enumerate}
\item \url{https://classroom.google.com/c/NTIzMTY2NTI4Njc0}
\item \url{https://classroom.google.com/c/NTIzMTY2NjExMTM2}
\item \url{https://classroom.google.com/c/NjE2MjE1NzIyMjA5}
\end{enumerate}

\vskip.15in
\noindent\textbf{Course Description:} The primary purpose of this course is to give new Civil Engineering (CE) students the necessary knowledge to make informed decisions regarding their future as students and later as professionals. It will help students answer the following important questions: \begin{enumerate}
	\item What is Civil Engineering? 
	\item Is Civil Engineering a viable career choice for me? 
	\item What challenges would I be facing as a CE student? 
	\item What is expected of me as a future Civil Engineer? 
\end{enumerate}

\vskip.15in
\noindent\textbf{Course Outcomes:}  At the end of this course, the student must be able to \begin{enumerate}
	\item define and classify Civil Engineering as a profession; 
	\item summarize the role played by Civil Engineers throughout the development of human civilization; 
	\item identify and evaluate the differences among various CE specializations; 
	\item recognize the challenges of being a Civil Engineering student in Mindanao State University – General Santos City 
	\item recognize the challenges of being a professional Civil Engineer. 
\end{enumerate}

\vskip.15in
%%%%%%%%%%%%%%%%%%%%%%%%%%%%%%%%%%
%DO THIS IN A SEPARATE FILE, THEN INCLUDE IN THIS IN THE MAIN TEX FILE
%%%%%%%%%%%%%%%%%%%
\noindent \textbf{Tentative Course Outline:}
\begin{center} 

\begin{tabular}{|c|l|}
\hline
\textbf{Module} & \textbf{Contents}  \\
\hline
Module 1 & Civil Engineering as a Discipline  \\
\hline
Module 2 &Skills and Tools for the Civil Engineering Student \\
\hline
Module 3 &BSCE Curriculum \& Student Co-Curricular Learning\\
\hline
Module 4 &Legends, Milestone, and Landmarks \\
\hline
Module 5 &Engineering Ethics\\
\hline
Module 6 &Beyond the BSCE Degree \\
\hline
\end{tabular}
\end{center}
\newpage
\vspace*{.15in}
\noindent\textbf{Grading Policy:} 
\begin{center} \begin{minipage}{3.8in}
\begin{flushleft}
Attendance     \dotfill 20.00\%  \\
Assignment/Seatwork/Course Project      \dotfill 20.00\%  \\
Periodic Test      \dotfill 25.00\%  \\
Final Exam       \dotfill 35.00\%  \\
\end{flushleft}
\end{minipage}
\end{center}



\vskip.15in
\noindent\textbf{Course Policy:}  
\begin{itemize}
\item A grade of Incomplete (INC) will be given to students who fail to take exams with VALID reasons. If a student fails to provide a valid reason for not being able to take two (2) consecutive chapter exams on designated schedules, he/she will be dropped from the subject. 

\item No removal exams will be given. Students will be graded on their Attendance, Assignment/Seatwork/ Course Project, performance on two (2) Periodic Tests, and Final Exam. 

\item If a student is not able to take any exam, a special exam will only be given if proper documents (e.g. excuse letter, etc.) are presented, otherwise, his/her score will automatically be 0. 

\item All petitions regarding this grading system will be entertained 1 week before the passing of grades after which, all decisions made by the instructor are final. 

\end{itemize}

\vskip.15in
\noindent\textbf{Class Policy:}  
\begin{itemize}
\item Regular attendance is essential and expected.
\item If missing a class could not be avoided (e.g. being sick, personal emergencies, etc.), an excuse letter is required from the student stating explicitly the reason for his/her absence and it should be signed by people who can vie for the veracity of the document (e.g. Doctor, parent, etc.). 
\item \textbf{TARDINESS}, likewise, will not be tolerated in this course. A fifteen (15) minute allowance/consideration will be given. Students who arrive beyond that time will be considered LATE. Three (3) consecutive tardiness will be considered as one (1) absence. 
\item Intellectual honesty is expected from all students. Any form of CHEATING will not be allowed. Students caught and proven to have participated in any of these wrongful acts shall automatically be given a FAILING GRADE (i.e. 5.00). Decisions made by the instructor regarding these matters are final.
\end{itemize}




%%%%%% THE END 
\end{document} 